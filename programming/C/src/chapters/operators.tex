\section{Operator precedence}


\subsection{Associativity and precedence}

% ref https://www.programiz.com/c-programming/precedence-associativity-operators
Associativity and precedence defined how operations are evaluated when there are multiple in a line.
\begin{itemize}
    \item \emphasis{Associativity} defines the order in which operations of the same precedence are evaluated in an expression. It can be left to right ($\rightarrow$) or right to left ($\leftarrow$).
    % ref https://en.cppreference.com/w/c/language/eval_order
    \item \emphasis{Precedence}  determines the grouping of terms in an expression and decides how an expression is evaluated. Certain operators have higher precedence than others.
\end{itemize}


\subsection{Precedence table}

% ref https://www.programiz.com/c-programming/precedence-associativity-operators
If more than one operators are involved in an expression, C language has a predefined rule of priority for the operators. This rule of priority of operators is called operator precedence.

% also https://www.ibm.com/support/knowledgecenter/en/SSLTBW_2.3.0/com.ibm.zos.v2r3.cbclx01/preeval.htm
%\TODO[copy \url{https://docs.microsoft.com/en-us/cpp/c-language/precedence-and-order-of-evaluation?view=vs-2019} below]

% ref for table https://aticleworld.com/operator-precedence-and-associativity-in-c/
\begin{tabular}{cclc} \toprule
    Rank & Operator & Type of operation & Associativity \\ \midrule
    1 & \texttt{()}  & Parentheses or function call &  \large{\ding{224}}  \\
    1 & \texttt{[]} & Brackets or array subscript & \large{\ding{224}} \\
    1 & \texttt{.} & Dot or member selection operator & \large{\ding{224}}\\
    1 & \texttt{->} & Arrow operator & \large{\ding{224}} \\
    1 & \texttt{++} or \texttt{---} & Postfix increment/ decrement & \large{\ding{224}} \\ \midrule
    2 & \texttt{++} or \texttt{---} & Prefix increment/ decrement & \large{\revdingarrow} \\
    2 & \texttt{+} or \texttt{-} & Unary plus or minus & \large{\revdingarrow} \\
    2 & \texttt{!}, \texttt{\~} & not operator, bitwise complement & \large{\revdingarrow} \\
    2 & \texttt{(type)}, e.g. \texttt{(double) 2} & type cast & \large{\revdingarrow} \\
    2 & \texttt{*}  & Indirection or dereference & \large{\revdingarrow} \\
    2 & \texttt{\&} & address of & \large{\revdingarrow}\\
    2 & \texttt{sizeof} & Determine size in bytes & \large{\revdingarrow} \\ \midrule
    3 & \texttt{* . \%} & Multiplication, division, modulus & \large{\ding{224}} \\ \midrule
    4 & \texttt{+ -} & Addition and subtraction & \large{\ding{224}} \\ \midrule
    5 & \texttt{<< >>} & Bitwise left shift and right shift & \large{\ding{224}}\\ \midrule
    6 & \texttt{< <=} & relational less/ less than or equal to & \large{\ding{224}} \\
    6 & \texttt{> >=} & relational greater/ greater than or equal to & \large{\ding{224}}\\ \midrule
    7 & \texttt{== !=} & relational equal or not equal to & \large{\ding{224}} \\ \midrule
    8 & \texttt{\&\&} & bitwise AND & \large{\ding{224}} \\ \midrule
    9 & \texttt{\^} & bitwise XOR & \large{\ding{224}} \\ \midrule
    10 & \texttt{|} & bitwise OR & \large{\ding{224}} \\ \midrule
    11 & \texttt{\&\&} & Logical AND & \large{\ding{224}} \\ \midrule
    12 & \texttt{||} & Logical OR & \large{\ding{224}}\\ \midrule
    13 & \texttt{?:} & Ternary operator & \large{\revdingarrow} \\ \midrule
    14 & \texttt{=} & Assignment & \large{\revdingarrow} \\ 
    14 & \texttt{+= -=} & Add/ subtract and assign & \large{\revdingarrow} \\
    14 & \texttt{*= /=} & Multiply/ divide and assign & \large{\revdingarrow} \\
    14 & \texttt{\%= \&=} & Modulus and assign/ bitwise AND and assign & \large{\revdingarrow} \\
    14 & $\hat{}\ $\texttt{= |=} & Bitwise XOR/ bitwise OR and assign & \large{\revdingarrow} \\
    14 & \texttt{<<= >>=} & Shift left/ shift right and assign & \large{\revdingarrow} \\ \midrule
    15 & \texttt{,} & comma operator\footnote{\url{https://stackoverflow.com/a/52558}} & \large{\ding{224}}  \\ \bottomrule
\end{tabular}

One in the table above that is not used often is the comma. (\texttt{,}).
\begin{takeaway}
Comma operator returns the rightmost operand in the expression and evaluates the rest, rejecting their return value.
\end{takeaway}

\begin{exmp}
\marginnote{It is no way implied that the following are good code practices, they simply test the understanding of operators!}
The left column shows some examples of the \texttt{,} operator and the right their output. 
\begin{multicols}{2}

\begin{lstlisting}[language=c]
#include <stdio.h>

int main(int argc, char *argv[])
{
	int i;
	i = 1, 2, 3;	
	printf("%d\n", i);
}
\end{lstlisting}

\columnbreak 
\; \\
\begin{verbatim}
1
\end{verbatim}

(= operation has highest precedence, therefore \texttt{i = 1} gets evaluated first. Then, \texttt{, 2, 3} gets evaluated, which does nothing.)
\begin{verbatim}
\end{verbatim}
\end{multicols}

\begin{multicols}{2}

\begin{lstlisting}[language=c]
#include <stdio.h>

int main(int argc, char *argv[])
{
	int i;
	i = (1, 2, 3);	
	printf("%d\n", i);
}
\end{lstlisting}

\columnbreak
\; \\
\begin{verbatim}
1
\end{verbatim}

(\texttt{()} have the highest priority, forcing what's inside them to get evaluated first, i.e. \texttt{1,2,3} to evaluate \texttt{3} (L to R). \texttt{3} gets assigned to \texttt{i}.)
\begin{verbatim}
\end{verbatim}
\end{multicols}

\begin{multicols}{2}

\begin{lstlisting}[language=c]
#include <stdio.h>

int main(int argc, char *argv[])
{
	int i = 1, 2, 3;	
	printf("%d\n", i);
}
\end{lstlisting}

\columnbreak
\; \\

(\texttt{=} has the highest priority, defining \texttt{i} as \texttt{1}. Since the \texttt{int} type is multiplicative, \texttt{int 2} and \texttt{int 3} will also be attempted to be declared -- compilation error.)
\begin{verbatim}
\end{verbatim}
\end{multicols}


\begin{multicols}{2}
\begin{lstlisting}[language=c]
#include <stdio.h>

int main(int argc, char *argv[])
{
	printf("%d bytes, %d bytes\n",
	    sizeof((double) (1,2,3)),
	    sizeof((int) (1.0, 2.0)));
}
\end{lstlisting}
\columnbreak

\; \\
\begin{verbatim}
8 bytes, 4 bytes
\end{verbatim}
(Outer parens first. Then inner parens, evaluating \texttt{3} and \texttt{2.0} respectively. Then type casts, evaluating \texttt{3.0} and \texttt{2} respectively. \texttt{sizeof} operates on each outer paren, returning 8 and 4 respectively.)

\end{multicols}




\begin{multicols}{2}

\begin{lstlisting}[language=c]
#include <stdio.h>

int main(int argc, char *argv[])
{
	int i = 0, j = 1, k = 2;
	int *p_i = &i, p_j = &j, p_k = &k;
	printf("%d %d %d\n", *p_i, *p_j, *p_k);
}
\end{lstlisting}
\columnbreak
\; \\
(compilation error -- \texttt{int} is multiplicative but dereference (\texttt{*}) is not. Pointer \texttt{p\_i} will be defined properly but \texttt{p\_j, p\_k} are just \texttt{int}. The correct way would be \texttt{*p\_j = \&j, *p\_k = \&k}.)
\end{multicols}

\clearpage
\begin{multicols}{2}
\begin{lstlisting}[language=c]
#include <stdio.h>

int main(int argc, char *argv[])
{
	int x = 0, y = 0;
	int i = (1, x++, ++y);
	printf("%d %d %d\n",
	    i, x++, ++y);
	printf("%d %d %d\n",
	    i = 1337, x, y);
}
\end{lstlisting}

\columnbreak
\; \\
\begin{verbatim}
1 1 2
1337 2 2
\end{verbatim}
(For explanation of the pre/post increment, see \ref{app:pre_post_increment}.)

\end{multicols}


\todo{add double loop example here... how long does it run?}
\begin{multicols}{2}
\begin{lstlisting}[language=c]
#include <stdio.h>

int main(int argc, char *argv[])
{
	int i, j = (printf("Hello?\n"),
			1337);
	printf("world!\n%d, %d\n",
			i, j);
}
\end{lstlisting}
\columnbreak
\; \\
\begin{verbatim}
Hello?
world!
0, 1337
\end{verbatim}
\end{multicols}


\begin{multicols}{2}
\begin{lstlisting}[language=c]
#include <stdio.h>

int main(int argc, char *argv[])
{
	int i;
	int arr[5] = {1, 2, 3, 4, 5};
	int *p_arr = arr; // same as &arr[0]

	// pr(++post) > pr(++pre) = pre(*deref))
	// ++p_arr; *p_arr (L to R) => 2
	printf("(1) %d\n", *++p_arr);
	// p_arr++, then *p_arr => 3
	printf("(2) %d\n", *p_arr++); // *
	// ++(*p_arr) => arr[2]++
	printf("(3) %d\n", ++*p_arr);
	// ++(*(p_arr++))
	printf("(4) %d\n", ++*p_arr++);
	for (i = 0; i < sizeof(arr)/sizeof(arr[0]);
			++i)
		printf("%d ", arr[i]);
	return 0;
}
\end{lstlisting}

\columnbreak
\; \\
\begin{verbatim}
(1) 2, 0xf9933074
(2) 2, 0xf9933078
(3) 4, 0xf9933078
(4) 5, 0xf993307c
1 2 5 4 5
\end{verbatim}

(1) Reading L to R, we evaluate \texttt{*(++p\_arr)}. \texttt{p\_arr} initially points at the 0-th element, so we print \texttt{2}.

(2) Post-inc has the highest priority, however evaluates AFTER the whole expression is evaluated, therefore we compute \texttt{*p\_arr} = 2, then increment the pointer.

(3) Equal priority, so we read L to R and evaluate \texttt{++}'s operand first. As we shifted the pointer before, result is \texttt{++(*p\_arr) = 3+1}. 

(4) Evaluated as \texttt{++(*p\_arr++) = *p\_arr++; ++p\_arr}
\end{multicols}

\clearpage
\begin{multicols}{2}
\begin{lstlisting}[language=c]
#include <stdio.h>

int main(int argc, char *argv[])
{
	int i = 0;
	int arr[5] = {1, 2, 3, 4, 5};

	printf("%d ", arr[++i]);
	printf("%d ", arr[i]);
	printf("%d ", arr[i++]);
	printf("%d \n", arr[i]);
	return 0;
}
\end{lstlisting}
\columnbreak
\begin{verbatim}
\; \\
2 2 2 3
\end{verbatim}
(Pre-increment always happens before the expressions have been evaluated and post after -- see \ref{app:pre_post_increment}.)
\end{multicols}


\begin{multicols}{2}
\begin{lstlisting}[language=c]
#include <stdio.h>

void main(){
   int intVar = 20, x;
   x = ++intVar, intVar++, ++intVar;
   printf("Value of intVar = \%d, x = \%d", intVar, x);
}
\end{lstlisting}
\columnbreak
\; \\
\begin{verbatim}
Value of intVar=23, x=21   
\end{verbatim}
(Since \texttt{=} operator has more precedence than \texttt{,} \texttt{=} operator will be evaluated first.
Here, \texttt{x = ++intVar, intVar++, ++intVar} so
\texttt{x = ++intVar} will be evaluated, assigning \texttt{21} to \texttt{x}. Then comma operator will be evaluated \cite{aptitudequestions}.)
\end{multicols}


\begin{multicols}{2}
\begin{lstlisting}[language=c]
#include <stdio.h>


void main()
{

	// (1)
	int x;
	x = (printf("AA") || printf("BB"));
	printf("%d\n", x);
	
	// (2)
	x = (printf("AA") && printf("BB"));
	printf("%d\n", x);
	
	// (3)
	x = printf("1") && printf("3") || 
		printf("3") && printf("7");
	printf("\n%d\n", x);
	
}
\end{lstlisting}


\columnbreak
\; \\
\begin{verbatim}
AA1
AABB1
13
1
\end{verbatim}

Adapted from \cite{aptitudequestions}. Keep in mind that \texttt{printf} returns the number of characters to be printed. 

(1) A logical OR (\texttt{ A || B}) condition checks whether A or B is true and as soon as one of them is true, exits. \texttt{printf("AA")} is true and returns \texttt{(int) true = 1} to \texttt{x}. In the end, prints \texttt{AA1}. (2) \texttt{A \&\& B} checks whether both A and B are true and if so returns \texttt{true}, therefore runs the two \texttt{printf} commands and returns \texttt{2 \&\& 2 = true} to \texttt{x}. (3) Similar to others, but evaluation stops when the OR (\texttt{||}), is hit, which is when the compiler understands the expression is true and returns 1 to \texttt{x}, therefore \texttt{13} and \texttt{1} are printed.
\end{multicols}



\begin{multicols}{2}


\begin{lstlisting}[language=c]


void main()
{   
	char var = 10;
	printf("var is = %d", ++var++);
}

\end{lstlisting}

\columnbreak
\; \\
\begin{verbatim}
lvalue required as increment operand
\end{verbatim}
Question adapted from \cite{aptitudequestions}. Will be evaluated as \texttt{(++var)++}. \texttt{++var} will yield \texttt{11}, which is an unnamed value -- unnamed values cannot be the operand of \texttt{++} \cite{aptitudequestions}.
\end{multicols}

\clearpage
\begin{multicols}{2}
\begin{lstlisting}[language=c]
#include <stdio.h>

void main()
{
   int a = 3, b = 2;
   a = a == b == 0;
   printf("%d,%d\n", a, b);
}
\end{lstlisting}

\columnbreak \; \\
\begin{verbatim}
1, 2
\end{verbatim}
\texttt{=} has the highest priority so the expression is evaluated as \texttt{ a = (a == b == 0)}. Then L to R as \texttt{a =((a == b) == 0)}, i.e. \texttt{a = (0 == 0)}, i.e. \texttt{a = 1} \cite{aptitudequestions}.
\end{multicols}
\end{exmp}


\subsubsection{Application of using operators to write concise code -- string manipulation}

A basic string library has been written to demonstrate how operators can be used for denser code. By using them, less buffer variables are needed. Remember that in C, strings are terminated by \texttt{'\arraybackslash 0'}, hence the conditions in the code. If precedence is correctly understood, then the code is easy to read too. Below are its functions. Prototypes are found in \texttt{sstr.h} and implementations at \texttt{sstr.c}.

\lstinputlisting[firstline=4,lastline=9,language=c,caption={String length implementation (\detokenize{src/sstrlib/sstr.c)}.}, label=src:sstrlen]{src/sstrlib/sstr.c}
The following diagram shows how \texttt{sstrlen("abcd")} works assuming \texttt{pSrc} points to an imaginary address \texttt{0x800}. First, we increment the pointer and then compare its value to 0.
\begin{verbatim}
+--+--+--+--+--+
|a |b |c |d |\0| (0x8001)
+--+--+--+--+--+
    |
    v
    True     
+--+--+--+--+--+
|a |b |c |d |\0| (0x8002)
+--+--+--+--+--+
       | 
       v
       True     
+--+--+--+--+--+
|a |b |c |d |\0| (0x8003)
+--+--+--+--+--+
          | 
          v
          True
+--+--+--+--+--+
|a |b |c |d |\0| (0x8004)
+--+--+--+--+--+
             |
             v
             False ------> 0x8004 - 0x8000
\end{verbatim}

\lstinputlisting[firstline=12,lastline=15,language=c,caption={String copy implementation  (\detokenize{src/sstrlib/sstr.c)}.}, label=src:sstrcpy]{src/sstrlib/sstr.c}

\lstinputlisting[firstline=18,lastline=22,language=c,caption={Reverse a string implementation  (\detokenize{src/sstrlib/sstr.c)}.}, label=src:sstrrev]{src/sstrlib/sstr.c}

\lstinputlisting[firstline=25,lastline=28,language=c,caption={Character to lowercase implementation  (\detokenize{src/sstrlib/sstr.c)}.}, label=src:sstrlower]{src/sstrlib/sstr.c}

\lstinputlisting[firstline=31,lastline=43,language=c,caption={Check palindrome implementation  (\detokenize{src/sstrlib/sstr.c)}.}, label=src:sstrpalin]{src/sstrlib/sstr.c}

\lstinputlisting[firstline=46,lastline=54,language=c,caption={Print string implementation  (\detokenize{src/sstrlib/sstr.c)}.}, label=src:sstrprint]{src/sstrlib/sstr.c}











%\url{https://stackoverflow.com/questions/15345396/precedence-of-dereference-and-postfix}\\
%\url{https://stackoverflow.com/questions/26902462/difference-between-argv-argv-argv-and-argv/26902610#26902610}\\
%\url{https://stackoverflow.com/questions/17251584/difference-between-int-i-1-2-3-and-int-i-1-2-3-variable-declaration-with}\\
%\url{https://www.youtube.com/watch?v=mhmnb80ZDBM}\\
%\url{https://www.2braces.com/c-questions/operators-questions-c-3}\\
%\url{https://www.includehelp.com/c/operators-aptitude-questions-and-answers.aspx}\\
%\url{https://stackoverflow.com/questions/4176328/undefined-behavior-and-sequence-points}\\
%\url{https://stackoverflow.com/questions/52550/what-does-the-comma-operator-do}\\
%\url{https://stackoverflow.com/questions/1613230/uses-of-c-comma-operator}\\
%\url{https://stackoverflow.com/questions/41611557/c-programming-comma-operator-within-while-loop}\\
%\TODO[example below]

%\url{https://docs.microsoft.com/en-us/cpp/c-language/precedence-and-order-of-evaluation?view=vs-2019}\\
%\url{https://www.includehelp.com/c/operators-aptitude-questions-and-answers.aspx}\\
%\url{https://www.2braces.com/c-questions/operators-questions-c-3}\\
%\url{https://www.sanfoundry.com/c-language-interview-questions-precedence-order-evaluation/}\\
%\url{https://gcc.gnu.org/onlinedocs/cpp/Operator-Precedence-Problems.html}\\
%\url{https://www.2braces.com/c-programming/c-operators-precedence}\\
%\url{http://web.deu.edu.tr/doc/oreily/java/langref/ch04_14.htm}\\
%\url{https://stackoverflow.com/questions/4176328/undefined-behavior-and-sequence-points}\\

